\nonstopmode{}
\documentclass[a4paper]{book}
\usepackage[times,inconsolata,hyper]{Rd}
\usepackage{makeidx}
\usepackage[utf8,latin1]{inputenc}
% \usepackage{graphicx} % @USE GRAPHICX@
\makeindex{}
\begin{document}
\chapter*{}
\begin{center}
{\textbf{\huge Package `BNDataGenerator'}}
\par\bigskip{\large \today}
\end{center}
\begin{description}
\raggedright{}
\item[Version]\AsIs{1.0}
\item[Date]\AsIs{2014-12-28}
\item[Title]\AsIs{Data Generator based on Bayesian Network Model}
\item[Author]\AsIs{Jae-seong Yoo}
\item[Maintainer]\AsIs{Jae-seong Yoo }\email{praster1@gmail.com}\AsIs{}
\item[Depends]\AsIs{R (>= 3.0.0)}
\item[Description]\AsIs{Previous tools suffer from serious trade-off between cost and complexity, restricting most studies relevant to Bayesian network to using only real data. To address such problem, a data generator based on Bayesian network model using R is built and introduced.}
\item[License]\AsIs{GPL (>= 2)}
\item[URL]\AsIs{}\url{http://www.github.com/praster1/BN\_Data\_Generator}\AsIs{}
\end{description}
\Rdcontents{\R{} topics documented:}
\inputencoding{utf8}
\HeaderA{BN\_Data\_Generator-package}{Data Generator based on Bayesian Network Model}{BN.Rul.Data.Rul.Generator.Rdash.package}
\aliasA{BN\_Data\_Generator}{BN\_Data\_Generator-package}{BN.Rul.Data.Rul.Generator}
\keyword{Bayesian Network}{BN\_Data\_Generator-package}
%
\begin{Details}\relax


\Tabular{ll}{
Package: & BN\_Data\_Generator\\{}
Type: & Package\\{}
Version: & 1.0\\{}
Date: & 2014-12-28\\{}
License: & GPL (>=2)\\{}
}
\end{Details}
%
\begin{Author}\relax
 Jae-seong Yoo <praster1@gmail.com> 
\end{Author}
%
\begin{References}\relax
Jae-seong Yoo, (2014), "A Study on Comparison of Bayesian Network Structure Learning Algorithms for Selecting Appropriate Models", M.S. thesis, Department of Statistics, Korea University, Seoul.
\end{References}
\inputencoding{utf8}
\HeaderA{big\_letters}{Letters built}{big.Rul.letters}
%
\begin{Description}\relax
The 26 or more lower-case letters of the Roman alphabet;
\end{Description}
%
\begin{Usage}
\begin{verbatim}
	big_letters(len)
\end{verbatim}
\end{Usage}
%
\begin{Arguments}
\begin{ldescription}
\item[\code{len}]  Get or set the length of letters. 
\end{ldescription}
\end{Arguments}
%
\begin{Author}\relax
 Jae-seong Yoo <praster1@gmail.com> 
\end{Author}
\inputencoding{utf8}
\HeaderA{BN\_Data\_Generator}{Data Generator based on Bayesian Network Model}{BN.Rul.Data.Rul.Generator}
%
\begin{Description}\relax
Data Generator based on Bayesian Network Model
\end{Description}
%
\begin{Usage}
\begin{verbatim}
	BN_Data_Generator(arcs_mat, input_Probs, n,// node_names = NULL, cardinalities = NULL)
\end{verbatim}
\end{Usage}
%
\begin{Arguments}
\begin{ldescription}
\item[\code{arcs\_mat}]  A matrix that determines the arcs.
\item[\code{input\_Probs}]  The conditional probabilities. 
\item[\code{n}]  Data size. 
\item[\code{node\_names}]  The names of each nodes. 
\item[\code{cardinalities}]  The cardinalities of each nodes. 
\end{ldescription}
\end{Arguments}
%
\begin{Author}\relax
 Jae-seong Yoo <praster1@gmail.com> 
\end{Author}
%
\begin{References}\relax
Jae-seong Yoo, (2014), "A Study on Comparison of Bayesian Network Structure Learning Algorithms for Selecting Appropriate Models", M.S. thesis, Department of Statistics, Korea University, Seoul.
\end{References}
\inputencoding{utf8}
\HeaderA{check\_cardinalities}{A checker of needs of input\_Prob follows cardinality}{check.Rul.cardinalities}
%
\begin{Description}\relax
A checker of needs of input\_Prob follows cardinality.
\end{Description}
%
\begin{Usage}
\begin{verbatim}
	check_cardinalities(arcs_mat, node_names = NULL, cardinalities = NULL)
\end{verbatim}
\end{Usage}
%
\begin{Arguments}
\begin{ldescription}
\item[\code{arcs\_mat}]  A matrix that determines the arcs. 
\item[\code{node\_names}]  The names of each nodes. 
\item[\code{cardinalities}]  The cardinalities of each nodes. 
\end{ldescription}
\end{Arguments}
%
\begin{Author}\relax
 Jae-seong Yoo <praster1@gmail.com> 
\end{Author}
\inputencoding{utf8}
\HeaderA{C\_M\_WO\_WC}{Correct, Missing, Wrongly Oriented, Wronglyy Corrected Arcs}{C.Rul.M.Rul.WO.Rul.WC}
%
\begin{Description}\relax
The existence of the known network structures allows us to define three important terms which indicate the performance of the algorithm (in terms of the number of graphical errors in the learnt structure).
\end{Description}
%
\begin{Usage}
\begin{verbatim}
	C_M_WO_WC(target_arcs_mat, learnt_arcs_mat)
\end{verbatim}
\end{Usage}
%
\begin{Arguments}
\begin{ldescription}
\item[\code{target\_arcs\_mat}]  A matrix of known network structure. 
\item[\code{learnt\_arcs\_mat}]  A matrix of learnt network structure. 
\end{ldescription}
\end{Arguments}
%
\begin{Value}
\begin{ldescription}
\item[\code{C (Correct Arcs)}] Edges present in the original network and in the learnt network structure.
\item[\code{M (Missing Arcs)}] Edges present in the original network but not in the learnt network structure.
\item[\code{WO (Wrongly Oriented Arcs)}] Edges present in the learnt network structure, but having opposite orientation when compared with the corresponding edge in the original network structure.
\item[\code{WC (Wrongly Corrected Arcs)}] Edges not present in the original network but included in the learnt network structure.
\end{ldescription}
\end{Value}
%
\begin{Author}\relax
 Jae-seong Yoo <praster1@gmail.com> 
\end{Author}
%
\begin{References}\relax
X.-w. Chen, G. Anantha, and X. Wang, (2006), An effective structure learning method for constructing gene networks, Bioinformatics, Vol. 22, 1367-1374.
\end{References}
\inputencoding{utf8}
\HeaderA{fromto\_to\_mat}{Convert from 'fromto' to 'matrix'}{fromto.Rul.to.Rul.mat}
%
\begin{Description}\relax
Convert from 'fromto' to 'matrix' that determines the arcs.
\end{Description}
%
\begin{Usage}
\begin{verbatim}
	fromto_to_mat(fromto, node_names)
\end{verbatim}
\end{Usage}
%
\begin{Arguments}
\begin{ldescription}
\item[\code{fromto}]  A matrix form structured 'fromto' that determines the arcs.
\item[\code{node\_names}]  The names of each nodes. 
\end{ldescription}
\end{Arguments}
%
\begin{Author}\relax
 Jae-seong Yoo <praster1@gmail.com> 
\end{Author}
\inputencoding{utf8}
\HeaderA{gen\_asia}{Asia (synthetic) data based on a model set by Lauritzen and Spiegelhalter}{gen.Rul.asia}
%
\begin{Description}\relax
Small synthetic data set from Lauritzen and Spiegelhalter (1988) about lung diseases (tuberculosis, lung cancer or bronchitis) and visits to Asia.
\end{Description}
%
\begin{Usage}
\begin{verbatim}
	gen_asia()
\end{verbatim}
\end{Usage}
%
\begin{Value}
\begin{ldescription}
\item[\code{D (dyspnoea)}] A two-level factor with levels yes and no.
\item[\code{T (tuberculosis)}] A two-level factor with levels yes and no.
\item[\code{L (lung cancer)}] A two-level factor with levels yes and no.
\item[\code{B (bronchitis)}] A two-level factor with levels yes and no.
\item[\code{A (visit to Asia)}] A two-level factor with levels yes and no.
\item[\code{S (smoking)}] A two-level factor with levels yes and no.
\item[\code{X (chest X-ray)}] A two-level factor with levels yes and no.
\item[\code{E (tuberculosis versus lung cancer/bronchitis)}] A two-level factor with levels yes and no.
\end{ldescription}
\end{Value}
%
\begin{Note}\relax
Lauritzen and Spiegelhalter (1988) motivate this example as follows:
Shortness-of-breath (dyspnoea) may be due to tuberculosis, lung cancer or bronchitis, or none of them, or more than one of them. A recent visit to Asia increases the chances of tuberculosis, while smoking is known to be a risk factor for both lung cancer and bronchitis. The results of a single chest X-ray do not discriminate between lung cancer and tuberculosis, as neither does the presence or absence of dyspnoea.
Standard learning algorithms are not able to recover the true structure of the network because of the presence of a node (E) with conditional probabilities equal to both 0 and 1. Monte Carlo tests seems to behvae better than their parametric counterparts.
\end{Note}
%
\begin{Author}\relax
 Jae-seong Yoo <praster1@gmail.com> 
\end{Author}
%
\begin{References}\relax
Lauritzen S, Spiegelhalter D (1988). "Local Computation with Probabilities on Graphical Structures and their Application to Expert Systems (with discussion)". Journal of the Royal Statistical Society: Series B (Statistical Methodology), 50(2), 157-224.
\end{References}
\inputencoding{utf8}
\HeaderA{is\_acyclic}{Acyclic graphs}{is.Rul.acyclic}
%
\begin{Description}\relax
This function checks for each node in a DAG whether backtracing arcs leading to it results in an "infinite recursion" error indicating that there actually is a cyclic part in the DAG (which then obviously seems not to be a DAG).
\end{Description}
%
\begin{Usage}
\begin{verbatim}
	is_acyclic(arcs_mat)
\end{verbatim}
\end{Usage}
%
\begin{Arguments}
\begin{ldescription}
\item[\code{arcs\_mat}]  A matrix that determines the arcs. 
\end{ldescription}
\end{Arguments}
%
\begin{Value}
A list with two elements. acyclic is a boolean indicating whether the DAG is acyclic (=TRUE) or contains a cyclic component (=FALSE). nodewise is a vector containing 1 boolean per node in the DAG, TRUE indicating that backtracing from this node does not lead to a cyclic component, FALSE indicating that backtracing from this node leads to a cyclic component.
\end{Value}
%
\begin{Author}\relax
 Jae-seong Yoo <praster1@gmail.com> 
\end{Author}
%
\begin{SeeAlso}\relax
\code{\LinkA{is\_DAG}{is.Rul.DAG}}
\end{SeeAlso}
\inputencoding{utf8}
\HeaderA{is\_DAG}{Directed acyclic graphs}{is.Rul.DAG}
%
\begin{Description}\relax
This function tests whether the given graph is a DAG, a directed acyclic graph.
\end{Description}
%
\begin{Usage}
\begin{verbatim}
	is_DAG(arcs_mat)
\end{verbatim}
\end{Usage}
%
\begin{Arguments}
\begin{ldescription}
\item[\code{arcs\_mat}]  A matrix that determines the arcs. 
\end{ldescription}
\end{Arguments}
%
\begin{Details}\relax
is\_dag checks whether there is a directed cycle in the graph. If not, the graph is a DAG.
\end{Details}
%
\begin{Value}
A logical vector of length one.
\end{Value}
%
\begin{Author}\relax
 Jae-seong Yoo <praster1@gmail.com> 
\end{Author}
%
\begin{SeeAlso}\relax
\code{\LinkA{is\_acyclic}{is.Rul.acyclic}}
\end{SeeAlso}
\inputencoding{utf8}
\HeaderA{make\_topology}{Bayesian Networks with varying topologies}{make.Rul.topology}
%
\begin{Description}\relax
Bayesian Networks with varying topologies (DAGs) with number of nodes.
\end{Description}
%
\begin{Usage}
\begin{verbatim}
	make_topology(nodes, topology = "Collapse", input_Probs = NULL, node_names = NULL, cardinalities = NULL)
\end{verbatim}
\end{Usage}
%
\begin{Arguments}
\begin{ldescription}
\item[\code{nodes}]  The number of nodes. 
\item[\code{topology}]  Geometric characteristic. 
\item[\code{input\_Probs}]  The conditional probabilities. 
\item[\code{node\_names}]  The names of each nodes. 
\item[\code{cardinalities}]  The cardinalities of each nodes. 
\end{ldescription}
\end{Arguments}
%
\begin{Details}\relax
The volume of the manifold is a geometric characteristic associated with the topology of Bayesian network. Each BN produces a different magnitude of the volume based on the DAG of Bayesian network.
Collapse, Line, Star, PseudoLoop, Diamond, Rhombus.
\end{Details}
%
\begin{Author}\relax
 Jae-seong Yoo <praster1@gmail.com> 
\end{Author}
%
\begin{References}\relax
Eitel J. M. L., (2008), An Information-geometric approach to learning Bayesian network topologies from data, Innovations in Bayesian Networks Studies in Computational Intelligence, Vol. 156, 187-217.
\end{References}
\inputencoding{utf8}
\HeaderA{mat\_to\_fromto}{Convert from 'matrix' to 'fromto'}{mat.Rul.to.Rul.fromto}
%
\begin{Description}\relax
Convert from 'matrix' to 'fromto' that determines the arcs.
\end{Description}
%
\begin{Usage}
\begin{verbatim}
	mat_to_fromto(arcs_mat)
\end{verbatim}
\end{Usage}
%
\begin{Arguments}
\begin{ldescription}
\item[\code{arcs\_mat}]  A matrix that determines the arcs that determines the arcs. 
\end{ldescription}
\end{Arguments}
%
\begin{Author}\relax
 Jae-seong Yoo <praster1@gmail.com> 
\end{Author}
\inputencoding{utf8}
\HeaderA{toss\_value}{Tossing a Cardinality}{toss.Rul.value}
%
\begin{Description}\relax
Sets up a sample space for the experiment of tossing a cardinality repeatedly with the outcomes "Values".
\end{Description}
%
\begin{Usage}
\begin{verbatim}
	toss_value(times, num_of_cases, makespace = FALSE)
\end{verbatim}
\end{Usage}
%
\begin{Arguments}
\begin{ldescription}
\item[\code{times}]  Number of tosses. 
\item[\code{num\_of\_cases}]  Cardinality. 
\item[\code{makespace}]  Logical. 
\end{ldescription}
\end{Arguments}
%
\begin{Value}
A data frame, with an equally likely probs column if makespace is TRUE.
\end{Value}
%
\begin{Note}\relax
It developed of 'tosscoin' function in prob package.
\end{Note}
%
\begin{Author}\relax
 Jae-seong Yoo <praster1@gmail.com> 
\end{Author}
%
\begin{Examples}
\begin{ExampleCode}
	toss_value(1, 3)
	toss_value(2, 3)
	toss_value(3, 4, makespace = TRUE)
\end{ExampleCode}
\end{Examples}
\printindex{}
\end{document}

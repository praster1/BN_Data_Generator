\vspace{1cm} {\small \ \indent
본 논문에서는 R의 bnlearn 패키지에서 제공하는 베이지안 네트워크 구조학습 알고리즘 간의 성능을 비교하였다.

베이지안 네트워크 구조 학습 결과에 대한 성능 평가는 score를 이용하는 방법과, 목표 네트워크와 학습된 네트워크를 서로 비교하는 방법이 있다. 본 논문에서는 이 두 가지 방법으로 알고리즘별 성능을 비교했을 때, 결과가 서로 다를 수 있음을 확인하였다.

Topology에 따른 Synthetic Data를 생성, 이에 대하여 알고리즘별 성능을 비교하여, 목표 네트워크의 형태에 따라 적합한 알고리즘 선택을 할 수 있도록 객관적인 방향을 제시하고자 하였다.

그동안 베이지안 네트워크 데이터 생성기가 매우 고가이거나, 공개된 툴도 매우 사용하기 까다로웠기 때문에, 베이지안 네트워크 관련 실증 연구는 사례 데이터를 이용한 경우가 대부분이었다. 이에 따라 Bayesian Network 모델을 바탕으로 R에서 데이터를 생성할 수 있는 생성기를 제작하여 공개하였다.
}
	\begin{figure}[h]
	\centering
		\includegraphics[height=50pt]{images/Topologies_PseudoLoop}
		\caption{Bayesian Network Topologies : Pseudo Loop}
	\end{figure}	

	% Pseudo Loop은, 우선 line 형태를 그린 뒤, 최상의 부모 node가 맨 마지막 자식 node에 종속되는 arc가 추가된 형태이다. 사실 loop는 아니지만, 얼핏 보면 loop처럼 보인다. (사실, 정말 loop가 만들어지면 더 이상 Bayesian Network가 아니다.)
	At first, drew a line form. And next, root node has depended on the very last child node. Then it called Pseudo Loop. Actually loop does not have, it looks like a loop at first glance. (In fact, actually when loop is created, no longer Bayesian Network is not it.)

\begin{figure}[!bhp]
	\centering
		\includegraphics[height=155pt]{images/Result_PseudoLoop}
		\caption{Summary for Comparison via Pseudo Loop}
	\end{figure}	

% Score 기준으로 비교했을 때는 TABU search가 좋은 성능을 보여주었지만, 목표 네트워크와 학습 네트워크를 비교했을 경우, 상대적으로 Hill-climbing이 line 형태에서 좋은 성능을 보여주었다.
Although TABU search when compared on the basis of exhibited good performance Score, when compared to the network and learning network objectives, relatively Hill-climbing showed good performance in the line form.

% sample size가 1000개일 때는 TABU search의 C의 개수가 개선되지 못했지만, sample size가 커지면 node 개수가 많을 때 C의 개수가 크게 향상되는 모습을 보여주었다. 특히 sample size 증가에 따른 M, WO, WC의 개수 감소가 두드러졌다. 그럼에도 불구하고 Hill-climbing의 성능에 미치지는 못하였다.
When the sample size is 1000, the number of C by TABU search has not been improved. I shows how the number of C when sample size is larger, then greatly improved. In particular M, WO, and WC with increasing sample size, reduced noticeable. And yet, did not win to the performance of Hill-climbing.

% 모든 알고리즘이,Sample size가 커짐에 따라 WO, WC 개수가 크게 줄어드는 모습이 나타났다.
All algorithms while sample size increases, WO and WC is greatly reduced.
	
	\begin{figure}[p]
	\centering
		\includegraphics[height=500pt]{images/04_PseudoLoop_Score}
		\caption{Comparison of scores via Pseudo Loop}
	\end{figure}	

	\begin{figure}[p]
	\centering
		\includegraphics[height=500pt]{images/04_PseudoLoop_Arcs}
		\caption{Comparison of correct arcs via Pseudo Loop}
	\end{figure}	

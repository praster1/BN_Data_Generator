~~~~A Hill-climbing is a greedy search on the space of the directed graphs. The optimized implementation uses score caching, score decomposability and score equivalence to reduce the number of duplicated tests.

\begin{center}\rule[0.5ex]{0.9\columnwidth}{1pt}\end{center}

\textbf{Algorithm.} \underline{The Hill-climbing(HC) Algorithm}

\begin{enumerate}
	\item \textbf{Current}: Make$\_$Node(Initial State)
	
	\item \textbf{While}
	
	~~~~\textbf{Neighbor}: a highest-valued successor of Current.State
	
	~~~~\textbf{If} Neighbor.Value $<$ Current.Value \textbf{Then}
	
	~~~~~~~~\textbf{Return} Current.State
	
	~~~~\textbf{End If}
	
	~~~~Current $\leftarrow$ Neighbor
	
	\textbf{End While}
\end{enumerate}

\begin{center}\rule[0.5ex]{0.9\columnwidth}{1pt}\end{center}

It is simply a loop that continually moves in the direction of increasing value. The algorithm does not maintain a search tree, so the data structure for the current node only needs to record the state and the value of the objective function. Hill-climbing does not look beyond the immediate neighbors of the current state. This resembles trying to find the top of Mount Everest in a thick fog while suffering from amnesia. (Russell S. J. and Norvig P., 2009)
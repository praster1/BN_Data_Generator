~~~~A modified Hill-climbing is able to escape local optima by selecting a network that minimally decreases the score function.

A variant of Hill-climbing called TABU search has gained popularity (Fred W. G. and Manuel L., 1997). This algorithm maintains a TABU list of $k$ previously visited states that cannot be revisited, as well as improving efficiency when searching graphs. This list allows the algorithm to escape from some local minima.

\begin{center}\rule[0.5ex]{0.9\columnwidth}{1pt}\end{center}

\textbf{Algorithm.} \underline{The TABU Search Algorithm}

\begin{enumerate}
	\item Choose $x \in X$ to start the precess.
	
	\item Find $x' \in N(x)$ such that $f(x') < f(x)$.
	
	\item If no such $x'$ can be found, $x$ is the local optimum and the method stops.
	
	\item Otherwise, designate $x'$ to be the new $x$ and go to 2.
\end{enumerate}

\begin{center}\rule[0.5ex]{0.9\columnwidth}{1pt}\end{center}

TABU search begins in the same way as ordinary local or neighborhood search, proceeding iteratively from one point solution to another until a chosen termination criterion is satisfied. Each $x \in X$ has an associated neighborhood $N(x) \subset X$, and each solution $x' \in N(x)$ is reached from $x$ by an operation called 'move'.

As an initial point of departure, we may contrast TABU search with a simple descent method where the goal is to $\min f(x)$ (or a corresponding ascent method where the goal is to $\max f(x)$). Such method only permits moves to neighbor solutions that improve the current objective function value and ends when no improving solutions can be found. A pseudo-code of a generic descent method is presented in 'Algorithm'. The final $x$ obtained by a descent method is called a local optimum, since it is at least as good or better than all solutions in its neighborhood. The evident shortcoming of a descent method is that such a local optimum in most cases will not be a global optimum, i.e., it usually will not minimize $f(x)$ over all $x \in X$.